%%% File encoding is ISO-8859-1 (also known as Latin-1)
%%% You can use special characters just like ä,ü and ñ

% Chapter without numbering but with appearance in the Table of Contents
% \addchap is a command from KOMA-Script
\addchap{Abstract}

This thesis investigates the theoretical data structure presented in the "Dynamic Integer Sets with Optimal Rank, Select, and Predecessor Search" paper, authored by Pătrașcu and Thorup. Concretely, we follow the paper closely up to chapter 3.1 (inclusive), implementing the data structure and its algorithms, evaluating its soundness. The authors claim that their proposal solves the dynamic predecessor problem optimally.

The aforementioned paper presents a data structure denoted \textit{dynamic fusion node}, which improves on the fusion node authored by Fredman and Willard and published in \cite{fredman1993surpassing}.
Fusion nodes compute rank and select queries in $O(\log_w n)$ time, but updates may entail recomputing the instance variables, making such operation to take polynomial time. It is stated that, for this reason, fusion trees only solve the static predecessor problem \cite{nelsonjelanilec2}. Pătrașcu and Thorup address this by improving on how the keys in the set are compressed, allowing them to be computed and queried in $O(1)$ time.




We bring together the techniques mentioned by the authors in \cite{patrascu2014dynamic} and implement them as they are needed to enable the correct implementation of the 

bittricks
In their publication, the authors mention that "the most interesting aspect" of their data structure is that "it supports all operations in constant time for sets of size $n = w^{O(1)}$". This claim derives from the fact that the technical details of the data structure are all based on word operations, which in the word-RAM model take constant time. this is a bit \texttt{1}{\ttfamily 1} \texttt{algorithm}{\ttfamily algorithm}

We explore the techniques of the theoretical paper, analyzing 

make a bullet list of the things I've done and then expand from there.

We will see in detail the techniques used by the authors to enable th

\paragraph*{Keywords:} Algorithms, Dynamic Predecessor Problem, Integer Sets, Key Compression, Bit-wise Operations, Most Significant Set Bit, Parallel Comparison, Dynamic Fusion Node.

from the meeting with holger:
\begin{itemize}
    \item
    much shorter version of the introduction
    
    \item
    2 or 3 very sharp paragraphs tailored for an expert in the field. say what 
    
    \item
    for people who want to get a very quick overview of what you're doing
    
    \item
    don't have to explain the background there. only on an extremely high level. don't need to say "oh a fusion tree is this"
\end{itemize}


% the last thing to write. it entails
\begin{itemize}
    \item
    Research problem and objective\\
    What practical or theoretical problem does this research respond to\\
    What research question did you aim to answer\\
    key verbs: test/investigate/analyze\\
    \item
    Methods\\
    Write in the past tense
    \item
    Key results or arguments
    \item
    Conclusion\\
    What is my answer to the question.
\end{itemize}