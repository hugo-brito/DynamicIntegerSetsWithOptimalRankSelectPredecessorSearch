% Chapter without numbering but with appearance in the Table of Contents
% \addchap is a command from KOMA-Script
\addchap{Abstract}

This thesis investigates the theoretical data structure presented in the «Dynamic Integer Sets with Optimal Rank, Select, and Predecessor Search» paper, authored by Pătrașcu and Thorup.
The authors claim that their proposal solves the dynamic predecessor problem optimally \cite{patrascu2014dynamic}.
Concretely, we follow the paper closely up to chapter 3.1 (inclusive), implementing the data structure and its algorithms, discussing and evaluating the running times.
The data structure they propose denoted \textit{Dynamic Fusion Node}, is made possible due to the smart use of techniques such as bitwise tricks, word-level parallelism, key compression and wildcards (denoted "don't cares" in this report and also by Pătrașcu and Thorup), which for sets of size $n = w^{O(1)}$ and in the word-RAM model take $O(1)$ time.
By using the \textit{Dynamic Fusion Node} in the implementation of a $k$-ary tree, thus enabling the sets to be of arbitrary size, the running times are $O(\log_w n)$, proven by Fredman and Saks to be optimal \cite{beame1999optimal}.

Using the \cite{patrascu2014dynamic} as our primary reference, and after framing the topic within its context (Chapters~\ref{sec:introductionChapter}), we explore the essential tools and algorithms used by Pătrașcu and Thorup in their proposal (Chapter~\ref{sec:backgroundChapter}).
Section~\ref{sec:IntegerSets} shows different ways to solve the predecessor problem.

Section~\ref{sec:summaryOfTechniques} builds and explores a library of relevant functions and algorithms used by the \textit{Dynamic Fusion Node}. These have also been implemented and mentioned further, of Chapter~\ref{sec:implementationsChapter}.

Based on the theoretical algorithms presented in \cite{patrascu2014dynamic}, the implementation is presented in iterative steps, starting from a naive way up to the insertion method while using all the algorithms and techniques described up to that point. This is presented in Chapter~\ref{sec:implementationsChapter}.

We validate the implemented algorithms and data structures with correctness tests. These appear described in Chapter~\ref{sec:validationChapter}.

Chapter~\ref{sec:conclusionChapter} concludes the project, leaving some remarks and suggestions for future work.
Up to the point where this project ends, Pătrașcu and Thorup's proposal seems sound, as we have been able to implement their data structure proposal using only $O(1)$ operations of the word RAM model. In Section~\ref{sec:futureWork} we highlight the required steps to conclude the implementation.

The Java programming language is used throughout the project, and the resulting code is publicly available on the GitHub repository sited at \url{https://github.com/hugo-brito/DynamicIntegerSetsWithOptimalRankSelectPredecessorSearch} and subject to an MIT License.

\paragraph*{Keywords:} Algorithms, Dynamic Predecessor Problem, Integer Sets, Key Compression, Bitwise Operations, Most Significant Set Bit, Word-level Parallelism, Dynamic Fusion Node.

% from the meeting with holger:
% \begin{itemize}
%     \item
%     much shorter version of the introduction
    
%     \item
%     2 or 3 very sharp paragraphs tailored for an expert in the field. say what 
    
%     \item
%     for people who want to get a very quick overview of what you're doing
    
%     \item
%     don't have to explain the background there. only on an extremely high level. don't need to say "oh a fusion tree is this"
% \end{itemize}


% % the last thing to write. it entails
% \begin{itemize}
%     \item
%     Research problem and objective\\
%     What practical or theoretical problem does this research respond to\\
%     What research question did you aim to answer\\
%     key verbs: test/investigate/analyze\\
%     \item
%     Methods\\
%     Write in the past tense
%     \item
%     Key results or arguments
%     \item
%     Conclusion\\
%     What is my answer to the question.
% \end{itemize}