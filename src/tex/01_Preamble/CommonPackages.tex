% Input encoding is 'utf-8'
% CTAN: http://www.ctan.org/pkg/inputenc
% 
% A newer package is available - you may look into:
% \usepackage[x-iso-8859-1]{inputenc}
% CTAN: http://www.ctan.org/pkg/inputenx
\usepackage[utf8]{inputenc}

% Font Encoding is 'T1' -- important for special characters such as Umlaute ü or ä and special characters like ñ (enje)
% CTAN: http://www.ctan.org/pkg/fontenc
\usepackage[T1]{fontenc}

% ttfamily font for code snippets and typewriter font (code snippets and links).
\usepackage[scaled=.85]{sourcecodepro}

% Language support for 'english' (alternative 'ngerman' or 'french' for example)
% CTAN: http://www.ctan.org/pkg/babel
\usepackage[english]{babel}
\babelhyphenation[english]{every-where}

% Doing calculations with LaTeX units -- needed for the vertical line in the footer
% CTAN: http://www.ctan.org/pkg/calc
\usepackage{calc}

% Extended graphics support 
% There is also a package named 'graphics' - watch out!
% CTAN: http://www.ctan.org/pkg/graphicx
\usepackage{graphicx}

% Extendes support for floating objects (tables, figures), adds the [H] placing option (\begin{figure}[H]) which palces it "Here" (without any doubt).
% CTAN: http://www.ctan.org/pkg/float
\usepackage{float}

% Extended color support
% I use the command \definecolor for example. 
% Option 'Table': Load the colortbl package, in order to use the tools for coloring rows, columns, and cells within tables.
% CTAN: http://www.ctan.org/pkg/xcolor
\usepackage[table]{xcolor} 

% Nice tables
% CTAN: http://www.ctan.org/pkg/booktabs
\usepackage{booktabs}
% Ability to have multirow columns on tables:
\usepackage{multirow}

%centering cell content in tables
\usepackage{array}

% Better support for ragged left and right. Provides the commands \RaggedRight and \RaggedLeft. 
% Standard LaTeX commands are \raggedright and \raggedleft
% http://www.ctan.org/pkg/ragged2e
\usepackage{ragged2e}

% Create function plots directly in LaTeX
% CTAN: http://www.ctan.org/pkg/pgfplots
\usepackage{pgfplots}
\pgfplotsset{compat=1.11}

% math symbols
\usepackage{amsmath}

% theorems, lemmas and corollaries:
\newtheorem{theorem}{Theorem}%[section]
\newtheorem{corollary}{Corollary}[theorem]
\newtheorem{lemma}[theorem]{Lemma}

% Code formatting
\usepackage{listings}
\lstset{
  language=java,
  extendedchars=true,
  basicstyle=\small\ttfamily,
  showstringspaces=false,
  showspaces=false,
  numbers=left,
  numberstyle=\footnotesize,
  numbersep=9pt,
  tabsize=4,
  breaklines=true,
  showtabs=false,
  frame=single,
  extendedchars=false,
  inputencoding=utf8,
  captionpos=b
}

% graphs and images
\usepackage{tikz}
\usetikzlibrary{positioning}
\usetikzlibrary{shapes}
\usetikzlibrary{fit}

% more than 1 image inside the same figure environment
\usepackage{subfig}

% for folder structure drawing
\usepackage{dirtree}

% checkmark symbol
\def\checkmark{\tikz\fill[scale=0.4](0,.35) -- (.25,0) -- (1,.7) -- (.25,.15) -- cycle;} 


\usepackage{mathtools}
% \makeatletter
% \newcommand*{\coloneqq}{\mathrel{\vcenter{\baselineskip0.5ex \lineskiplimit0pt
%                      \hbox{\scriptsize.}\hbox{\scriptsize.}}}%
%                      =}